%% LyX 1.3 created this file.  For more info, see http://www.lyx.org/.
%% Do not edit unless you really know what you are doing.
\documentclass[size=10pt,letterpaper,landscape,english,nopsheader,style=fyma,display=slidesnotes]{powerdot}
\usepackage[T1]{fontenc}
\usepackage[latin1]{inputenc}

\makeatletter

%%%%%%%%%%%%%%%%%%%%%%%%%%%%%% LyX specific LaTeX commands.
\providecommand{\LyX}{L\kern-.1667em\lower.25em\hbox{Y}\kern-.125emX\@}
\newcommand{\lyxarrow}{\leavevmode\,$\triangleright$\,\allowbreak}

%%%%%%%%%%%%%%%%%%%%%%%%%%%%%% User specified LaTeX commands.
\usepackage{listings}
\pdsetup{%
  lf=left footer,
  rf=right footer
}

\usepackage{babel}
\makeatother
\begin{document}

\title{A \LyX{} sample}


\author{Mael Hill�reau \and Hendri Adriaens \and Christopher Ellison}


\date{\today}

\maketitle

\lyxend\lyxslide{My first slide}

We use a \texttt{Slide} style (magenta color, see above this text
into \LyX{} window) for the title of each slide of a section or document.


\lyxend\lyxwideslide{My wide slide}

\label{toc}But we may also use wide slides (green color, see above
this text into \LyX{} window).

Let's have a look at the TOC:
\medskip{}

\tableofcontents{}


\lyxend\section{This is the first section}


\lyxend\lyxslide[toc=This slide has...]{This slide has a very very long title}

This slide's title is different in the TOC, see slide \pageref{toc}
or at the left of all slides in this section.


\lyxend\lyxnote{Important note!}

Don't forget to use \texttt{Insert}\lyxarrow{}\texttt{Short title}
in order to specify slide options.


\lyxend\lyxslide{Using \texttt{\textbackslash{}pause}}

Let's make a pause! \pause

\ldots{} and then continue.


\lyxend\section{Second section}


\lyxend\lyxslide{Using \texttt{enumerate}}

Let's enumerate things:\pause

\begin{enumerate}[type=1]
\item The first thing;\pause
\item The second thing;\pause
\item The third thing;\pause
\item The final thing!
\end{enumerate}

\lyxend\lyxslide{Using \texttt{itemize}}

We may also make some items and nest them:\pause

\begin{itemize}[type=1]
\item <2> The first;
\item <3> The second;

\begin{itemize}[type=1]
\item <4> One,
\item <5> Two,
\item <6> Three,
\end{itemize}
\item <7> The last one.
\end{itemize}

\lyxend\lyxemptyslide[toc=,randomdots]{The empty slide!}

\bigskip{}
\begin{center}This slide is empty\ldots{} \end{center}

\pause

\begin{center}It uses \texttt{EmptySlide} style for its title (see
cyan color into \LyX{} window).\end{center}

\pause

\begin{center}You may also have noticed the use of \texttt{randomdots}
slide option which\\
allows displaying those multicolor dots into the slide background.\end{center}


\lyxend\lyxslide[method=direct]{Example of \LaTeX{} source code}

Let's now use the \texttt{listings} package in order to typeset some
source code. Here's the \texttt{\textbackslash{}HelloWorld} \LaTeX{}
command:

\lstset{language=[LaTeX]TeX}
\begin{lstlisting}
\newcommand{\HelloWorld}{Hello World!}
\end{lstlisting}

Note that this slide uses the \texttt{method=direct} option (see its
short title options box above into \LyX{} window).


\lyxend\lyxslide[method=file]{Verbatim material with overlays}

Here's \texttt{HelloWorld()} C function:

\lstset{language=C}
\begin{lstlisting}
void HelloWorld() {
  printf("Hello World!\n");
}
\end{lstlisting}

\pause Unlike the previous one, this slide must use \texttt{method=file}
as it employs two overlays.


\lyxend\lyxslide{The end.}

Don't forget to put an \texttt{EndSlide} style at the end of the document
(see the black symbol below into \LyX{} window).

Have fun!


\lyxend{}
\end{document}
