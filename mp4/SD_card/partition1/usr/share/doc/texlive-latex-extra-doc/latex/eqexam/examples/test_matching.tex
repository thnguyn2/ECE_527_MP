\documentclass{article}
\usepackage{amsmath}
\usepackage{graphicx}
%\usepackage[forpaper,pointsonleft]{eqexam}
\usepackage[forpaper,pointsonleft,nosolutions]{eqexam}
%\usepackage[online,forpaper,pointsonleft,answerkey]{eqexam}

\title[T1]{Test 1}
\author{D. P. Story}
\subject[Eq]{EqExam}
\date{Spring \the\year}
\keywords{Test 1, Section 001}

\university
{%
      NORTHWEST FLORIDA STATE COLLEGE\\
          Department of Mathematics
}
\email{storyd@nwfsc.edu}

\hfuzz = .7pt

\begin{document}
\maketitle


\begin{exam}{Part1}

\begin{instructions}
Solve each of the problems without error. If you make an error,
points will be subtracted from your total score.
\end{instructions}


\begin{problem*}[\auto]
Match each of names on the left, with the corresponding names on the right.

\begin{multicols}{2}
\begin{parts}
\item\PTs*{2}\fillin{.5in}{\ref{george}} Washington
\item\PTs*{2}\fillin{.5in}{\ref{john}} Adams
\item\PTs*{2}\fillin{.5in}{\ref{thomas}} Jefferson

\columnbreak

\renewcommand{\thepartno}{\Alph{partno}}
\partsformat{\Alph{partno}.}

% The above redefinitions are reset when we leave the group, when \end{parts}
% is completed. If the original definitions are needed to be reset before then
% execute the following commands: \defaultthepartno, \defaultpartsformat

% \foritem{a} resets the partno counter back to 0, and displays (a), which
% has been redefined to A.

\foritem{a}\label{john} John

\item\label{thomas} Thomas

\item\label{george} George

\end{parts}
\end{multicols}
\end{problem*}

\end{exam}

\end{document}
