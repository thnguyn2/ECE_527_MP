\documentclass[10pt]{newlfm}

\newlfmP{leftmargintopdist=1.65in,%
dateskipbefore=12pt,dateskipafter=12pt,%
stdmemo,Avery5160,useenvlab,setuplabel,unprleft=16pt,unprtop=16pt,%
addrtoskipbefore=4pt,headermarginskip=10pt,%
addrtoskipafter=2pt,addrfromskipbefore=24pt,addrfromskipafter=24pt,Blankheader,%
letrh=mastc,addrt=PIT,addrf=CCR,sigskipcolumn=7pt,sigskipbefore=8pt,addrfromright,orderdatefromto}%
\sigacross{3}
\LUheader{\framebox{testa}}\siglist{sLvBa,sLvBb,sLvBc}
\RUheader{testc}
\CUheader{testb}
\speciala{Test1}{2}{2}\spacespb{6pt}{6pt}
\specialb{Test2}{4}{2}
\usepackage{tabls}
\re{Fonts and music}
\regarding{Orders for trucks}
\begin{document}
\begin{newlfm}
  In terms of the relationships between font
  descriptions and musical typography, it seems to
  me that only in the case of the early 1800s
  should we think of such a really odd notion.

\end{newlfm}
\end{document}

%%% Local Variables: 
%%% mode: latex
%%% TeX-master: t
%%% End: 
