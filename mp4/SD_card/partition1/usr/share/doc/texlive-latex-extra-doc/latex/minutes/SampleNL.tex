\def\minfileversion{V1.8c}     %^^Aof minutes.sty
\def\minfiledate{2009/05/17}   %^^Aof minutes.sty
%%
%% \CharacterTable
%%  {Upper-case    \A\B\C\D\E\F\G\H\I\J\K\L\M\N\O\P\Q\R\S\T\U\V\W\X\Y\Z
%%   Lower-case    \a\b\c\d\e\f\g\h\i\j\k\l\m\n\o\p\q\r\s\t\u\v\w\x\y\z
%%   Digits        \0\1\2\3\4\5\6\7\8\9
%%   Exclamation   \!     Double quote  \"     Hash (number) \#
%%   Dollar        \$     Percent       \%     Ampersand     \&
%%   Acute accent  \'     Left paren    \(     Right paren   \)
%%   Asterisk      \*     Plus          \+     Comma         \,
%%   Minus         \-     Point         \.     Solidus       \/
%%   Colon         \:     Semicolon     \;     Less than     \<
%%   Equals        \=     Greater than  \>     Question mark \?
%%   Commercial at \@     Left bracket  \[     Backslash     \\
%%   Right bracket \]     Circumflex    \^     Underscore    \_
%%   Grave accent  \`     Left brace    \{     Vertical bar  \|
%%   Right brace   \}     Tilde         \~}
%%
\begin{Notulen}{Titel van Nederlandse Notulen}
\ondertitel{Ondertitel}
\voorzitter{Voorzitter}
\notulist{Notulist}
\deelnemer{Deelnemer}
\gast{Gasten}
\bijeenkomstdatum{25.\ Dezember 1999}
\beginbijeenkomst{10:00}
\eindbijeenkomst{20:00}
\locatie{Amsterdam}
\cc{Verenigingsleden}
\afwezig[Afwezig met bericht]{afwezig zonder bericht}
%%\afwezigBericht{}
%%\afwezigZonderBericht{}
\notulenkop

\topic{Onderwerp een}%<-- hier Tagesordnungspunkt einfuegen
\subtopic{Deelonderwerp bij onderwerp een}%<-- Unterpunkt
Tekst bij punt van orde.
%%
\topic{Termijnen en Taken}
\subtopic{Tijdsschema}
\termijn{2000/12/24}[10:00]{Kerst voordag}
\termijn{2000/12/24}{Kerstavond}
\termijn{2000/12/24}[20:00]{Bescherung}[Zal mijn wens vervuld worden?]
\termijn*{2000/12/25}{Kerstmis (zonder kalender invoer)}

\topic{Taken}
\aktie{Wie}{Wat}
\aktie*{Iemand doet iets}
\aktie*[Vandaag]{Iemand zal vandaag iets doen}
\aktie[Voldaan]{Huidige toestand}[Gisteren]{Iemand doet iets}

\nieuweKolom[][1]%nodig vanwege een fout
\begin{Geheim}
\topic{Geheim Onderwerp}
Deze tekst is geheim en kan met de optie \texttt{Secret} afgedrukt
worden.
\end{Geheim}

\subtopic{Na geheime tekst}
Als een onderwerp niet in het hoofdonderwerp  "`Geheim"' ligt, dan
word het afdrukken van de aeheime tekst onderdrukt.
\geheim{Een klein geheimpje}
\newcols

\extrapunt{Notulen buiten standaard notulen}
\subtopic{Deelonderwerp}

\topic{Bijlage}
\bijlage{Bijlage met twee bladzijden}{2}

\topic{Afspraken en Stemmingen }
\subtopic{Enkele stemmingen}
\stemming{Korte stemming}{1}{2}{3}

Stemming met resultaat:\par
\stemming{Korte stemming}{1}{2}{3}[Resultaat]

\subtopic{Meerdere stemmingen in volgorde}
\begin{Stemming}
\stemming{Stemming een}{1}{2}{3}
\stemming{Stemming twee}{1}{2}{3}[Beslissing]
\stemming{Stemming drie}{1}{2}{3}
\end{Stemming}

\subtopic{Besluiten}
\besluitonderwerp{Thema}{Titel van het onderwerp}
\besluit{Thema}{Besluit genomen}
\besluit*{Besluit zonderThema}[Lange tekst over het besluit]
\end{Notulen}
\endinput
%%
%% End of file `SampleNL.tex'.
