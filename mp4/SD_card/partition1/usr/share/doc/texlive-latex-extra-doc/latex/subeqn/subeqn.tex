%%
%% This is file `subeqn.tex',
%% generated with the docstrip utility.
%%
%% The original source files were:
%%
%% subeqn.dtx  (with options: `sample')
%% 
%% Copyright (C) 1999-2004 Johannes Braams. All rights reserved.
%% 
%% This file was generated from file(s) of the subeqn package.
%% -----------------------------------------------------------
%% 
%% It may be distributed and/or modified under the
%% conditions of the LaTeX Project Public License, either version 1.3
%% of this license or (at your option) any later version.
%% The latest version of this license is in
%%   http://www.latex-project.org/lppl.txt
%% and version 1.3 or later is part of all distributions of LaTeX
%% version 2003/12/01 or later.
%% 
%% This work has the LPPL maintenance status "maintained".
%% 
%% The Current Maintainer of this work is Johannes Braams.
%% 
%% This file may only be distributed together with a copy of the
%% subeqn package. You may however distribute the subeqn package
%% without such generated files.
%% 
%% The list of all files belonging to the subeqn package is
%% given in the file `manifest.txt.
%% 
%% The list of derived (unpacked) files belonging to the distribution
%% and covered by LPPL is defined by the unpacking scripts (with
%% extension .ins) which are part of the distribution.
%% File: `subeqn.dtx'
%% Copyright (C) 1999-2004
%%     Donald Arsenau (asnd at reg.triumf.ca),
%%     Johannes Braams (TeXniek at braams.cistron.nl)
%%
\ProvidesFile{subeqn.tex}
              [2004/04/15 v2.0b subnumbering of equations]
\documentclass{article}
\usepackage{subeqn}

\begin{document}
This is an example ot the use of the \texttt{subeqations} package.
\begin{equation}
  \label{a}
  a^2 + b^2 = c^2
\end{equation}
Now we start sub-numbering.
\begin{subequations}
  \label{b}
  \begin{equation}
    \label{b1}
    d^2 + e^2 = f^2
  \end{equation}
  We can refer to equation~\ref{a}, \ref{b} and~\ref{b1}.
  \begin{equation}
    \label{b2}
    g^2 + h^2 = i^2
  \end{equation}
  This was equation~\ref{b2}.
  \begin{eqnarray}
    \label{c}
    x &=& y+z\label{c1}\\
    u &=& v+w\label{c2}
  \end{eqnarray}
  This was expression~\ref{c}, consisting of parts~\ref{c1}
  and~\ref{c2}.
\end{subequations}

Now lets start a \textsf{subeqnarray} environment.
\begin{subeqnarray}
  \label{d}
  x &=& y+z\label{d1}\\
  u &=& v+w\label{d2}
\end{subeqnarray}
This was equation~\ref{d}, with parts~\ref{d1} and~\ref{d2}.
\end{document}
\endinput
%%
%% End of file `subeqn.tex'.
