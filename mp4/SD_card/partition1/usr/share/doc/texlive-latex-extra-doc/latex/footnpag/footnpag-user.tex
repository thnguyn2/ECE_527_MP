% $DocId: footnpag-user.tex,v 2.1 1995/08/04 00:18:33 schrod Exp $
%------------------------------------------------------------

%
% user manual/requirement definition for footnpag package
%
% [LaTeX]
% (history at end)


% This file is either a subdocument of the style doc or a document on
% its own. In the former case it's a chapter, in the latter it's a
% ``normal'' LaTeX progltx document.
%     If it's a subdocument, this file will be included after
% \begin{document}. We can detect this: \document redefines
% \documentclass to be \@twoclasseserror. Then we also have to define
% how this document is ended: Either by \endinput or by an additional
% revision log.
% Of course, this test works only if LaTeX 2e is used for processing.

\expandafter\ifx \csname @twoclasseserror\endcsname \documentclass

        \let\endSubDocument=\endinput

        \chap What's this package for?.

\else

        \let\endSubDocument=\relax

        \documentclass{progltx}

        \usepackage{footnpag-doc}       % document-specific markup
        \usepackage{a4-9}               % Tschichold's A4 layout

        \nofiles                        % no crossreferences used

        \begin{document}

        \title{The \texttt{footnpag} Package}
        \author{Joachim Schrod%
            \thanks{Email: \texttt{schrod@iti.informatik.th-darmstadt.de}}%
            }

        \RCS $DocDate: 1995/08/04 00:18:33 $
        \date{%
            \RCSDocDate\\[3pt] % LaTeX Error: Paragraph terminated too early
            (Revision \RCSStyleRevision{} of \texttt{footnpag.sty})%
            }

        \maketitle

        \sect

\fi





This package is appropriate for numbering footnotes separately on each
page. It may be used with all standard document classes (and I assume
with all other well written ones). `Numbering' here does not only mean
supplying Arabic numbers, if your class or another package sets up
appropriate symbols, they are used instead.

You just have to use the |footnpag| package, the rest will happen
automagically. You will need two \LaTeX{} runs, as with cross
references or citations.


\sect \textsl{Known Problems}.

\medskip

\noindent Unlike with references, \LaTeX{} will not issue a warning if
a footnote number is incorrect due to new page breaks. Thus run
\LaTeX{} always another time if you want a final document.

This is scheduled to change in the next revision.


\sect An auxiliary file is used, named \textit{jobname}|.fot|. This is
not necessary by itself, it would have been better to use the
|aux| file. (Actually, this package was written for plain \TeX{}
originally, there is no |aux| file there.)

This is also scheduled to change in the next revision. In fact, it's
the same problem like the previous one: When I use the |aux| file,
\LaTeX{}'s warning mechanism can be utilized.


\sect There is no possibility to change the number that's used for the
first footnote on a page.

Some people like to start their footnotes with~`0', there exists even
a package for that named |zero|. |footnpag| should cooperate with that
package.

If one uses symbols as footnote markers, the first footnote is marked
with an asterisk. One might want to use a dagger instead, that's not
possible currently.

Most probably this feature will be made available via the new
``keyword-value'' option scheme, as you'll find it in the |graphicx|
package.




\endSubDocument


%%%%%%%%%%%%%%%%%%%%%%%%%%%%%%%%%%%%%%%%%%%%%%%%%%%%%%%%%%%%%%%%%%%%%%

\vskip \PltxPreSectSkip

\begin{rcslog}
$DocLog: footnpag-user.tex,v $
\Revision 2.1 1995/08/04 00:18:33 schrod
Made a \LaTeXe{} package from this style option.

User manual is a separate document now, that's better for
installation. Started to change the distribution into one that
conforms to the `supported bundle guidelines.'

\end{rcslog}



\end{document}

% LocalWords:  footnpag jobname fot graphicx
