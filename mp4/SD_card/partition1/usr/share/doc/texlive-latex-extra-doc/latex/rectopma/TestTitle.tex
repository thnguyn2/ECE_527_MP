% THIS IS BOTH A MINIMAL USER-MANUAL 
% OF THE PACKAGE  rectopma.sty
% AND  AN EXAMPLE OF ITS  USE 
%
% File  :   TestTitle.tex 
% Author:   Battista Benciolini
% E-mail:   <Battista.Benciolini@ing.unitn.it>
% Date  :   January  2002
% See file  rectopma.sty  for more information
%
\documentclass[a4paper,10pt]{article}
\usepackage{rectopma}   
%
\date{ }
\title{Test of the package \texttt{rectopma} \intitlebreak and 
suggestions for its use \intitlebreakvs (I need a long title)\thanks{
Comments are welcome !}}
\author{B.Benciolini\thanks{e-mail: battista.benciolini@ing.uitn.it}
\and  No Second Author\thanks{No-Where Institute}}
%
\SaveTopMatter
%
\begin{document}
\maketitle 
%
\section{Introduction}
The package \verb+rectopma+ makes it possible to reuse the main 
content of \verb+\title+ and \verb+\author+ in different parts of a 
document. 
It is also possible to force linebreaks in the title with a command 
that is only active inside the top-matter, not when the title is 
re-printed elsewhere.
%
\section{Instruction}
The new commands \verb+\intitlebreak+ and \verb+\intitlebreakvs+
(vs= vertical skip) are used to force a line break in the title 
that disappears when the title itself is re-used outside the top matter 
of the paper.
The content of \verb+\title+ and \verb+\author+ must be saved with the 
command \verb+\SaveTopMatter+ before the action of \verb+\maketitle+ 
and can be reprinted with the new commands \verb+\SavedAuthor+
and \verb+\SavedTitle+. When they are reprinted the names of the 
various authors are simply separated by commas and the content of 
\verb+\thanks+ is ignored.
%
\section{Examples}
The authors of this paper are listed here after: \SavedAuthor. This 
is obtained with \verb+\SavedAuthor+. It is also possible to reprint 
the title, by means of  \verb+\SavedTitle+, and the result is: 
\SavedTitle. The title can be printed with a different style, as in:
\textbf{\SavedTitle}, obtained  with  \verb+\textbf{\SavedTitle}+.
%
\end{document}
\endinput
%
% End of file TestPaper.tex 
