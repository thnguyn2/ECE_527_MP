<x-flowed>
Thank you for promulgating the newlfm.cls letter/fax/memo class.  I am 
gearing up to use it in my daily work, and am also using it to try to 
improve my LaTeX and (nearly-nonexistent) TeX skills. 

I am having the following problem in the "Regarding" section, which I 
believe is due to a bug.

In the attached testmemo.tex file, when I use the option "stdletter", 
the "Regarding" line prints out "Testing"; however, when I use the 
option "stdmemo", the "Re" line prints out "---BLANK---".      As near 
as I can tell, the  @re@line   variable is not being set correctly, but 
with my basic and rudimentary TeX skills, I'm at a loss to figure out how.

Attached are extracd.tex, letrinfo.tex, and my test case, 
testmemo.tex.   I am using newlfm v 8.3, downloaded this past week from 
CTAN. 

Any help would be appreciated.  Thanks again.


   James
James K. Gruetzner, Ph.D.

Resolution: Fixed - cannot duplicate

---------------------------------------------------------------------
> > I've been using newlfm over quite a few incarnations, and have 
> > discovered a bug/feature on which you may be able to comment.
> >
> > While it may not be your preferred or recommended means of 
> > operation, in my role as secretary of an organisation, I sometimes 
> > choose to place all the letters arising from a particular meeting in


> > the one file (eg Letters_2004_10_29.tex).  What seems to happen is 
> > that when the file is processed, I get the first letter then its 
> > envelope, then the second letter and the envelopes for the FIRST and


> > second letters, then the third letter and the envelopes for the 
> > FIRST, SECOND, and third letters, end so on.  I think this behaviour


> > started about the time that you added the required functiuonality 
> > from the geometry package.  As far as I can tell, my file is not 
> > bizarre, just a series of \begin{newlfm} ... \end{newlfm} 
> > environments with the necessary addresses and phrase customisations 
> > etc.  Likewise, the same effect occurred with a form letter just 
> > recently so I suspect a glitch in the
>
> > interface between newlfm and envlab.
> >
> > I may have misunderstood some parameter settings, or may have an old

Fixed
----------------------------------------------------------------
Dear Dr. Thompson,

I have just begun using your newlfm package and have already found it
highly useful. Thank you very much for making it available to us.

I am writing today to inform you that newlfm may have a minor bug. The
newlfm that I am using is of version "2004/11/02 v8.3". It seems that
this version ignores the memonoto, memonofrom, and memono
options. Looking at the code of newlfm, I think that "true" in lines
513, 515, and 517 should be replaced with "false".


  513: \DeclareOption{memonofrom}   {\setboolean{@memo@e}{true}}%
                                                          ^^^^
  514: \define@key{ov}{memonofrom}[true]{\iffixt{#1}{@memo@e}}%
  515: \DeclareOption{memonoto}     {\setboolean{@memo@g}{true}}%
                                                          ^^^^
  516: \define@key{ov}{memonoto}[true]{\iffixt{#1}{@memo@g}}%
  517: \DeclareOption{memonore}     {\setboolean{@memo@f}{true}}%
                                                          ^^^^
  518: \define@key{ov}{memonore}[true]{\iffixt{#1}{@memo@f}}%


Thanks again for your offering the wonderful newlfm.

-----------------------------------
Fixed 


-----------------------------------

I just started using 2004/11/02 v8.3 of newlfm.  Nice job!


I'm having a small problem though.  I have specified a custom header
with graphics and that is working properly.  If I try to specify the
Blankheader option to turn it off, the file fails to format and I get
an error.  I have a feeling that I'm not specifying this option
properly and I couldn't find it in any of your examples.


How do I invoke this option?


Thanks,


Rick Zaccone
Computer Science Department
Bucknell University

----
FIxed

%%% Local Variables: 
%%% mode: latex
%%% TeX-master: "d"
%%% End: 
