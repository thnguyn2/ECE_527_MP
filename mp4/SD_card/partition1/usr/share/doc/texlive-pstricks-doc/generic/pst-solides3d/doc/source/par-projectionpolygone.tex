\section {Polygones}

L'objet \Cadre{polygone} permet de d�finir un polygone. Sous sa forme
usuelle, on utilise l'argument \verb+args+ pour sp�cifier la liste des
points~: 
\Cadre{[object=polygone,args=$A_0$ $A_1$ \ldots $A_n$]}

Il existe d'autres m�thodes pour d�finir un polygone 2d. L'argument
\Cadre{definition}, coupl� � l'argument \Cadre{args} permet d'utiliser
les diff�rentes m�thodes support�es~:

\begin{itemize}

%% syntaxe : pol u  --> pol'
\item \Cadre {[definition=translatepol]} ; 
\verb+args=+$pol$ $u$. Translat� du polygone $pol$  par le vecteur
$\vec u$

%% syntaxe : pol u  --> pol'
\item \Cadre {[definition=rotatepol]} ; 
\verb+args=+$pol$ $I$ $\alpha $. Image du polygone pol par la rotation
de centre $I$ et d'angle $\alpha $

%% syntaxe : pol I alpha  --> pol'
\item \Cadre {[definition=hompol]} ; 
\verb+args=+$pol$ $I$ $\alpha $. Image du polygone $pol$ par
l'homoth�tie de centre $I$ et de rapport $\alpha $.

%% syntaxe : pol I  --> pol'
\item \Cadre {[definition=sympol]} ; 
\verb+args=+$pol$ $I$. Image du polygone $pol$ par
la sym�trie de centre $I$.

%% syntaxe : pol D  --> pol'
\item \Cadre {[definition=axesympol]} ; 
\verb+args=+$pol$ $d$. Image du polygone $pol$ par
la sym�trie axiale de droite $d$.

\end{itemize}

Dans l'exemple ci-dessous, on d�finit, on nomme et on trace le
polygone de sommets $(1,0)$, $(-3, 1)$, $(0, 2)$ puis on trace en bleu
son image par la rotation de centre $(-1,0)$ et d'angle $-45$. Enfin,
on repr�sente le translat� du polygone d'origine par le vecteur
$(2,-2)$, et ce en incorporant directement du code jps dans l'argument
\verb+[definition=]+. 

\begin{multicols}{2}

\begin{pspicture}(-3,-3)(4,3.5)%
\psframe*[linecolor=blue!50](-3,-3)(4,3.5)
\psset{lightsrc=50 20 20,viewpoint=50 30 15,Decran=60}
\psset{solidmemory}
\psSolid[object=grille,
   base=-3 0 -3 3,
   linewidth=0.5\pslinewidth,linecolor=gray,]
%% definition du plan de projection
\psSolid[object=plan,
   definition=equation,
   args={[1 0 0 0] 90},
   base=-3.2 3.2 -2.2 2.2,
   name=monplan,
   planmarks,
]
\psset{plan=monplan}
\psSolid[object=plan,
   args=monplan,
   linecolor=gray!40,
   plangrid,
   action=none,
]
\psProjection[object=polygone,
   args=-1 0 -3 1 0 2,
   name=P,
]
\psProjection[object=polygone,
   definition=rotatepol,
   linecolor=blue,
   args=P -1 0 -45,
]
%% du code jps dans la definition
\psProjection[object=polygone,
   definition={2 -2 addv} papply,
   fillstyle=hlines,hatchcolor=yellow,
   linestyle=dashed,
   args=P,
]
\composeSolid
\axesIIID(4,2,2)(5,4,3)
\end{pspicture}
\columnbreak

\begin{gbar}
\begin{verbatim}
\psProjection[object=polygone,
   args=-1 0 -3 1 0 2,
   name=P,]
\psProjection[object=polygone,
   definition=rotatepol,
   linecolor=blue,
   args=P -1 0 -45,]
%% du code jps dans la definition
\psProjection[object=polygone,
   definition={2 -2 addv} papply,
   fillstyle=hlines,hatchcolor=yellow,
   linestyle=dashed,
   args=P,]
\end{verbatim}
\end{gbar}
\end{multicols}
