\setlongtables
\begin{longtable}{@{}>{\ttfamily}l@{}>{\ttfamily}l>{\ttfamily}l>{\RaggedRight\arraybackslash}X@{}}
\caption{R�sum� des param�tres de l'extension \texttt{pst-labo}}\label{tab:pst-labo:Parameter}\\
\emph{\textrm{Nom}} & \emph{\textrm{Valeur}}  & \emph{\textrm{D�faut}} & \emph{Commentaire}\\\hline
\endfirsthead
\emph{\textrm{Nom}} & \emph{\textrm{Valeur}}  & \emph{\textrm{D�faut}} & \emph{Commentaire}\\\hline
\endhead
%\multicolumn{3}{c}{\CMD{TubeEssais}}\\\hline

\Loption{glassType} & tube|ballon| &  tube & D�finit le type de \\
                    & becher|erlen|&       & verrerie\\
                    & \rlap{flacon|fioleJauge} \\
\Loption{bouchon} & \Larga{false|true} & false & Ferme la verrerie par un bouchon\\
\Loption{pince} & \Larga{false|true} & false & Pince en bois\\
%\Loption{fioleJaugee} & \Larga{false|true} & false&\\
\Loption{tubeDroit} & \Larga{false|true} & false & R�frig�rant � air\\
\Loption{tubeCoude} & \Larga{false|true} & false & Tube coud� � $90^\circ$\\ 
\Loption{tubeCoudeU} & \Larga{false|true} & false& Tube en U\\
\Loption{tubeCoudeUB} & \Larga{false|true} & false & Tube en U � associer
uniquement � la verrerie de type \verb+ballon+ ou \verb+erlen+\\
\Loption{tubeRecourbe} & \Larga{false|true} & false & \\
\Loption{tubeRecourbeCourt} & \Larga{false|true} & false & Lorsqu'il n'y a pas
de bec Bunsen\\
\Loption{tubePenche} & \Larga{$-65 \ldots 65$} & 0 & Angle d'inclinaison\\
\Loption{doubletube} & \Larga{false|true} & false & Pour le d�gagement gazeux
sans chauffage\\ 
%
\Loption{etiquette} & \Larga{false|true} & false & \\
\Loption{Numero} & \Larga{Text} & \{\} & Num�ro plac� sur l'�tiquette (voir l'option \verb+etiquette+)\\
%\multicolumn{3}{c}{\CMD{ChauffeTube}}\\\hline
\Loption{tubeSeul} & \Larga{false|true} & false & Environnement
\verb+pspicture+ large ou �troit\\
\Loption{becBunsen} & \Larga{false|true} & true & Avec ou sans bec Bunsen\\
%\Loption{tubedegagamentDroit} & \Larga{false|true} &false&\\% nicht vorhanden
\Loption{barbotage} & \Larga{false|true} & false & Place un tube � essai
secondaire pour r�cup�rer les d�gagements gazeux.\\
\Loption{substance}      & \Larga{Macro} & \CMD{relax} & \CMD{pstBullesChampagne}, \CMD{pstFilaments}, 
                                              \CMD{pstBilles}, \CMD{pstBULLES}, \CMD{pstClous}, \CMD{pstCuivre}\\
\Loption{solide}         & \Larga{Macro} & \CMD{relax} & \CMD{pstTournureCuivre}, \CMD{pstClouFer}, 
                                              \CMD{pstGrenailleZinc}\\
\Loption{refrigerantBoulle} & \Larga{false|true} & false & Pour le chauffage � reflux\\
\Loption{recuperationGaz} & \Larga{false|true} & false & Dispositif de
r�cup�ration des gaz\\
%
%\Loption{reactifBecher} & \{\} & b�cher ou ballon ou flacon\\\hline
%\Loption{reactifBurette} & \{\} &Formule et/ou concentration du r�actif\\
\rlap{\Loption{couleurReactifBurette}}\\
                  & \Larga{Couleur} & \rlap{OrangePale} & \\
\rlap{\Loption{niveauReactifBurette}}\\
                  & 20 & \Larga{$0\ldots 25$} & 25~mL maximum\\
%\Loption{echelle} & \Larga{Wert} & 1 & echelle du sch�ma\\
\Loption{AspectMelange} & \Larga{Style} & \rlap{DiffusionBleue}\\
\Loption{CouleurDistillat} & \Larga{Couleur} & yellow & \\
%
\Loption{phmetre} & \Larga{false|true} & false & Place un pH-m�tre\\
\rlap{\Loption{agitateurMagnetique}}\\
                                    & \Larga{false|true} & true & \\
%
\Loption{aspectLiquide1} & \Larga{Style} & cyan & D�fini par la commande
\CMD{newpsstyle}\\ 
\Loption{aspectLiquide2} & \Larga{Style} & yellow & \emph{idem}\\
\Loption{aspectLiquide3} & \Larga{Style} & magenta & \emph{idem}\\
\Loption{niveauLiquide1} & \Larga{$0 \ldots 100$} & 50 &\\
\Loption{niveauliquide2} & \Larga{$0 \ldots 100$} & 0  & < niveauLiquide1\\
\Loption{niveauliquide3} & \Larga{$0 \ldots 100$} & 0 &  < niveauLiquide2\\
\end{longtable}


