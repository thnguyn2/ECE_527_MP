\setlongtables
\begin{longtable}{@{}>{\ttfamily}l@{}>{\ttfamily}l>{\ttfamily}l>{\RaggedRight\arraybackslash}X@{}}
\caption{Zusammenfassung aller Parameter f�r \texttt{pst-labo}}\label{tab:pst-labo:Parameter}\\
\emph{\textrm{Name}} & \emph{\textrm{Werte}}  & \emph{\textrm{Vorgabe}} & \emph{Anmerkungen}\\\hline
\endfirsthead
\emph{\textrm{Name}} & \emph{\textrm{Werte}}  & \emph{\textrm{Vorgabe}} & \emph{Anmerkungen}\\\hline
\endhead
%\multicolumn{3}{c}{\CMD{TubeEssais}}\\\hline

\Loption{glassType} & tube|ballon| &  tube & bezeichnet den Typ des \\
                    & becher|erlen|&       &  Glasgef��es\\
                    & \rlap{flacon|fioleJauge} \\
\Loption{bouchon} & \Larga{false|true} & false & Gef�� wird mit einem Korken versehen.\\
\Loption{pince} & \Larga{false|true} & false & Holzklammer\\
%\Loption{fioleJaugee} & \Larga{false|true} & false&\\
\Loption{tubeDroit} & \Larga{false|true} & false & Glasr�hrchen\\
\Loption{tubeCoude} & \Larga{false|true} & false & abgewinkeltes Glasr�hrchen\\ 
\Loption{tubeCoudeU} & \Larga{false|true} & false& doppelt abgewinkeltes Glasr�hrchen\\
\Loption{tubeCoudeUB} & \Larga{false|true} & false & verl�ngerte Ausf�hrung, nur f�r den 
						    Glastyp \verb+ballon+ oder \verb+erlen+\\
\Loption{tubeRecourbe} & \Larga{false|true} & false & \\
\Loption{tubeRecourbeCourt} & \Larga{false|true} & false & Anordnung ohne Bunsenbrenner\\
\Loption{tubePenche} & \Larga{$-65 \ldots 65$} & 0 & Kippwinkel\\
\Loption{doubletube} & \Larga{false|true} & false & pour d�gagement gazeux sans chauffage\\
%
\Loption{etiquette} & \Larga{false|true} & false & \\
\Loption{Numero} & \Larga{Text} & \{\} & Nummer f�r die Option \verb+etiquette+\\
%\multicolumn{3}{c}{\CMD{ChauffeTube}}\\\hline
\Loption{tubeSeul} & \Larga{false|true} & false & breite/schmale \verb+pspicture+-Box\\
\Loption{becBunsen} & \Larga{false|true} & true & mit/ohne Bunsenbrenner\\
%\Loption{tubedegagamentDroit} & \Larga{false|true} &false&\\% nicht vorhanden
\Loption{barbotage} & \Larga{false|true} & false & zus�tzliches Reagenzglas, durch Glasr�hrchen mit dem
                                                   eigentlichen Gef�� verbunden\\
\Loption{substance}      & \Larga{Makro} & \CMD{relax} & \CMD{pstBullesChampagne}, \CMD{pstFilaments}, 
                                              \CMD{pstBilles}, \CMD{pstBULLES}, \CMD{pstClous}, \CMD{pstCuivre}\\
\Loption{solide}         & \Larga{Makro} & \CMD{relax} & \CMD{pstTournureCuivre}, \CMD{pstClouFer}, 
                                              \CMD{pstGrenailleZinc}\\
\Loption{refrigerantBoulle} & \Larga{false|true} & false & pour  chauffage � reflux\\
\Loption{recuperationGaz} & \Larga{false|true} & false & Anordnung zum Auffangen entwichener Gase\\
%
%\Loption{reactifBecher} & \{\} & b�cher ou ballon ou flacon\\\hline
%\Loption{reactifBurette} & \{\} &Formule et/ou concentration du r�actif\\
\rlap{\Loption{couleurReactifBurette}}\\
                  & \Larga{Farbe} & \rlap{OrangePale} & \\
\rlap{\Loption{niveauReactifBurette}}\\
                  & 20 & \Larga{$0\ldots 25$} & Begrenzung auf 25mL\\
%\Loption{echelle} & \Larga{Wert} & 1 & echelle du sch�ma\\
\Loption{AspectMelange} & \Larga{Stil} & \rlap{DiffusionBleue}\\
\Loption{CouleurDistillat} & \Larga{Farbe} & yellow & \\
%
\Loption{phmetre} & \Larga{false|true} & false & pHMesser anzeigen\\
\rlap{\Loption{agitateurMagnetique}}\\
                                    & \Larga{false|true} & true & \\
%
\Loption{aspectLiquide1} & \Larga{Stil} & cyan & definiert als Teil von \CMD{newpsstyle}\ldots\\
\Loption{aspectLiquide2} & \Larga{Stil} & yellow & dito\\
\Loption{aspectLiquide3} & \Larga{Stil} & magenta & dito\\
\Loption{niveauLiquide1} & \Larga{$0 \ldots 100$} & 50 &\\
\Loption{niveauliquide2} & \Larga{$0 \ldots 100$} & 0  & < niveauLiquide1\\
\Loption{niveauliquide3} & \Larga{$0 \ldots 100$} & 0 &  < niveauLiquide2\\
\end{longtable}


