\section{\Index{Chamfering} a solid}

\psset{lightsrc=10 0 10,viewpoint=50 -20 30 rtp2xyz,Decran=50}
\begin{LTXexample}[width=6cm]
\psset{unit=0.5}
\begin{pspicture*}(-4,-4)(4,4)
\psSolid[object=cube,
   a=5,
   fillcolor=red]
\end{pspicture*}
\end{LTXexample}


\begin{LTXexample}[width=6cm]
\psset{unit=0.5}
\begin{pspicture*}(-4,-4)(4,4)
\psSolid[object=cube,
   a=5,
   fillcolor=red,
   chanfrein,
   chanfreincoeff=.6]
\end{pspicture*}
\end{LTXexample}

The option \Lkeyword{chanfrein} allows us to \Index{chamfer} a solid. This option
uses the key \Lkeyword{chanfreincoeff} (value $0.8$ by default) which indicates the
ratio $k$ with ($0<k<1$). This ratio is the one of a centre dilation with
the centre in the middle of the chosen face.

\psset{lightsrc=10 0 10,viewpoint=50 -20 30 rtp2xyz,Decran=30}
\begin{LTXexample}[width=6cm]
\psset{unit=0.5}
\begin{pspicture*}(-4,-4)(4,4)
\psSolid[object=dodecahedron,
   a=5,
   fillcolor=cyan]
\end{pspicture*}
\end{LTXexample}

\psset{lightsrc=10 0 10,viewpoint=50 -20 30 rtp2xyz,Decran=30}
\begin{LTXexample}[width=6cm]
\psset{unit=0.5}
\begin{pspicture*}(-4,-4)(4,4)
\psSolid[object=dodecahedron,
   a=5,
   fillcolor=cyan,
   chanfrein,
   chanfreincoeff=.8]
\end{pspicture*}
\end{LTXexample}
%\newpage

\endinput
