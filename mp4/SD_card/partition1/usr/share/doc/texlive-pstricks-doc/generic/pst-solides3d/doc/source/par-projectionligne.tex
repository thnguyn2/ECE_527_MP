\section {Lignes}

L'objet \Cadre{line} permet de d�finir une ligne bris�e. Sous sa forme
usuelle, on utilise l'argument \verb+args+ pour sp�cifier la liste des
points~: 
\Cadre{[object=line,args=$A_0$ $A_1$ \ldots $A_n$]}

On peut �galement d�finir les transform�es d'une ligne bris�e par une
translation, une rotation, une homoth�tie, etc\dots en reprenant les
op�rations disponibles sur les polygones.

\begin{multicols}{2}

\begin{pspicture}(-3,-3)(4,3.5)%
\psframe*[linecolor=blue!50](-3,-3)(4,3.5)
\psset{lightsrc=50 20 20,viewpoint=50 30 15,Decran=60}
\psset{solidmemory}
\psSolid[object=grille,
   base=-3 0 -3 3,
   linewidth=0.5\pslinewidth,linecolor=gray,]
%% definition du plan de projection
\psSolid[object=plan,
   definition=equation,
   args={[1 0 0 0] 90},
   base=-3.2 3.2 -2.2 2.2,
   name=monplan,
   planmarks,
]
\psset{plan=monplan}
\psSolid[object=plan,
   args=monplan,
   linecolor=gray!40,
   plangrid,
   action=none,
]
\psProjection[object=line,
   args=-1 0 -3 1 1 2,
   name=P,
]
\psProjection[object=line,
   definition=rotatepol,
   linecolor=blue,
   args=P -1 0 -45,
]
%% du code jps dans la definition
\psProjection[object=line,
   definition={2 -2 addv} papply,
   linestyle=dashed,
   args=P,
]
\composeSolid
\axesIIID(4,2,2)(5,4,3)
\end{pspicture}
\columnbreak

\begin{gbar}
\begin{verbatim}
\psProjection[object=line,
   args=-1 0 -3 1 0 2,
   name=P,]
\psProjection[object=line,
   definition=rotatepol,
   linecolor=blue,
   args=P -1 0 -45,]
%% du code jps dans la definition
\psProjection[object=line,
   definition={2 -2 addv} papply,
   linestyle=dashed,
   args=P,]
\end{verbatim}
\end{gbar}
\end{multicols}

