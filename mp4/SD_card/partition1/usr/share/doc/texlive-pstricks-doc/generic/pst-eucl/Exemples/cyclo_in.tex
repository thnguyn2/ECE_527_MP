\renewcommand{\NbPt}{11}\psset{linewidth=1.2\pslinewidth}
\pstGeonode[PointSymbol={*,none}, PointName={default,none}, PosAngle=180]{M}(0,1){O}
%% 4*pi=12.5663706144
\pstGeonode(12.5663706144,0){A}
\pstTranslation[PointSymbol=none, PointName=none]{M}{A}{O}[B]
\multido{\n=1+1}{\NbPt}{%
  \pstHomO[HomCoef=\n\space \NbPt\space 1 add div,
           PointSymbol=none, PointName=none]{O}{B}[O\n]
  \pstProjection[PointSymbol=none, PointName=none]{M}{A}{O\n}[P\n]
  %\pstCircleOA[linestyle=dashed, linecolor=red]{O\n}{P\n}
  \pstCurvAbsNode[PointSymbol=square, PointName=none,CurvAbsNeg=true]
    {O\n}{P\n}{M\n}{\pstDistAB{O}{O\n}}
  \ifnum\n=2%affichage du second cercle
    \bgroup
    \pstCircleOA{O\n}{M\n}
    \psset{linecolor=magenta, linewidth=1.5\pslinewidth}
    \pstArcnOAB{{O\n}}{P\n}{M\n}
    \ncline{O\n}{M\n}\ncline{P\n}{M}
    \egroup
  \fi}% fin du multido
\psset{linecolor=blue, linewidth=1.5\pslinewidth}
\pstGenericCurve[GenCurvFirst=M]{M}{1}{6}
\pstGenericCurve[GenCurvLast=A]{M}{6}{\NbPt}
%% juste pour la v�rification
%\parametricplot[linecolor=green, linewidth=.5\pslinewidth]{0}{12.5663706144}
%  {t t 3.1415926 div 180 mul sin sub 1 t 3.1415926 div 180 mul cos sub}
