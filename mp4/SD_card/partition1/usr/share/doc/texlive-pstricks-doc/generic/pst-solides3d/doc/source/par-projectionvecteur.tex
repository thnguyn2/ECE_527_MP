\section {Vecteurs}

\subsection {D�finition directe}

L'objet \Cadre{vecteur} permet de d�finir et tracer un vecteur. Sous
sa forme la plus simple, on utilise l'argument l'argument \verb+args+
pour d�finir ses coordonn�es $(X,Y)$ et on sp�cifie le point d'origine
o� tracer le vecteur en utilisant les valeurs $(x,y)$ dans la commande
\verb+\psProjection+ (ou �ventuellement un point nomm�).

Comme pour les points, on peut sauvegarder les coordonn�es d'un
vecteur en utilisant l'option \Cadre{name}.

\begin{multicols}{2}

\begin{pspicture}(-3,-3)(4,3.5)%
\psframe*[linecolor=blue!50](-3,-3)(4,3.5)
\psset{viewpoint=50 30 15,Decran=60}
\psset{solidmemory}
%% definition du plan de projection
\psSolid[object=plan,
   definition=equation,
   args={[1 0 0 0] 90},
   planmarks,
   name=monplan,
]
\psset{plan=monplan}
%% definition du point A
\psProjection[object=point,
   args=-2 0.75,
   name=A,
   text=A,
   pos=dl,
]
\psProjection[object=vecteur,
   linecolor=red,
   args=1 1,
   name=U,
](1,0)
\psProjection[object=vecteur,
   args=U,
   linecolor=blue,
](A)
\composeSolid
\axesIIID(4,2,2)(5,4,3)
\end{pspicture}

\columnbreak

\begin{gbar}
\begin{verbatim}
\psProjection[object=point,
   args=-2 0.75,name=A,
   text=A,pos=dl,]
\psProjection[object=vecteur,
   linecolor=red,
   args=1 1,name=U,
](1,0)
\psProjection[object=vecteur,
   args=U,
   linecolor=blue,
](A)
\end{verbatim}
\end{gbar}

\end{multicols}

\subsection {Autres d�finitions}

Il existe d'autres m�thodes pour d�finir un vecteur 2d. L'argument
\Cadre{definition}, coupl� � l'argument \Cadre{args} permet d'utiliser
les diff�rentes m�thodes support�es~:

\begin{itemize}

\item \Cadre {[definition=vecteur]} ; 
\verb+args=+ $A$ $B$. Le vecteur $\overrightarrow {AB}$ 

\item \Cadre {[definition=orthovecteur]} ; 
\verb+args=+ $u$. Un vecteur orthogonal � $\vec u$ et de m�me norme. 

\item \Cadre {[definition=normalize]} ; 
\verb+args=+ $u$. Le vecteur $\Vert \vec u \Vert ^{-1} \vec u$ si $\vec
u \neq \vec 0$, et $\vec 0$ sinon.

\item \Cadre {[definition=addv]} ; 
\verb+args=+ $u$ $v$. Le vecteur $\vec u + \vec v$

\item \Cadre {[definition=subv]} ; 
\verb+args=+ $u$ $v$. Le vecteur $\vec u - \vec v$

\item \Cadre {[definition=mulv]} ; 
\verb+args=+ $u$ $\alpha $. Le vecteur $\alpha \vec u$

\end{itemize}
