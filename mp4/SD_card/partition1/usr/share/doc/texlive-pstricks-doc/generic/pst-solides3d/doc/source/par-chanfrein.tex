\section {Chanfreiner un solide}

\begin{multicols}{3}
\begin{center}
\psset{unit=0.5}
\psset{lightsrc=10 0 10,viewpoint=50 -20 30 rtp2xyz,Decran=50}
\begin{pspicture*}(-4,-4)(4,4)
\psframe(-4,-4)(4,4)
\psSolid[object=cube,
   a=5,
   fillcolor=red,
]
\end{pspicture*}
\end{center}
\columnbreak
\begin{center}
\psset{unit=0.5}
\psset{lightsrc=10 0 10,viewpoint=50 -20 30 rtp2xyz,Decran=50}
\begin{pspicture*}(-4,-4)(4,4)
\psframe(-4,-4)(4,4)
\psSolid[object=cube,
   a=5,
   fillcolor=red,
   chanfrein,
   chanfreincoeff=.6,
]
\end{pspicture*}
\end{center}
\columnbreak
\begin{verbatim}
\psSolid[object=cube,
   a=5,
   fillcolor=red,
   chanfrein,
   chanfreincoeff=.6,
]
\end{verbatim}
\end{multicols}

L'option \Cadre{[chanfrein]} permet de chanfreiner un solide. Cette
option utilise l'argument \Cadre{[chanfreincoeff]} (valeur $0,8$ par
d�faut) qui indique le rapport $k$ � utiliser ($0<k<1$). Ce rapport
est celui d'une homoth�tie de cetre le centre de la face consid�r�e.

\begin{multicols}{3}
\begin{center}
\psset{unit=0.5}
\psset{lightsrc=10 0 10,viewpoint=50 -20 30 rtp2xyz,Decran=30}
\begin{pspicture*}(-4,-4)(4,4)
\psframe(-4,-4)(4,4)
\psSolid[object=dodecahedron,
   a=5,
   fillcolor=cyan,
]
\end{pspicture*}
\end{center}
\columnbreak
\begin{center}
\psset{unit=0.5}
\psset{lightsrc=10 0 10,viewpoint=50 -20 30 rtp2xyz,Decran=30}
\begin{pspicture*}(-4,-4)(4,4)
\psframe(-4,-4)(4,4)
\psSolid[object=dodecahedron,
   a=5,
   fillcolor=cyan,
   chanfrein,
   chanfreincoeff=.8,
]
\end{pspicture*}
\end{center}
\columnbreak
\begin{verbatim}
\psSolid[object=dodecahedron,
   a=5,
   fillcolor=cyan,
   chanfrein,
   chanfreincoeff=.8,
]
\end{verbatim}
\end{multicols}


%\newpage
