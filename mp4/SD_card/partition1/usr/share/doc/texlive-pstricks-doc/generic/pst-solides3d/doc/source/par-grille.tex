\section {La grille}

L'objet \Cadre{grille} permet d'obtenir un solide plan.
Le param�tre \Cadre{base=$x{min}$ ${x{max}}$ ${y{min}}$
${y{max}}$]} permet de sp�cifier la taille de la grille.

\begin{multicols}{2}
\psset{viewpoint=50 50 20 rtp2xyz,Decran=30}
\begin{pspicture}(-4.5,-1.5)(2.5,2.5)
\psframe(-3.5,-1.5)(2.5,2.5)
\psSolid[object=grille,
   base=0 4 -3 3,
   linecolor=gray](0,0,0)
\axesIIID(0,0,0)(4,3,3)
\end{pspicture}

\columnbreak

\begin{verbatim}
\psSolid[object=grille,
   base=0 4 -3 3,
   linecolor=gray](0,0,0)
\end{verbatim}
\end{multicols}

Le param�tre \Cadre{[ngrid=$n_1$ $n_2$]} permet de sp�cifier le
maillage de la grille. Si $n_2$ est absent, on consid�re que $n_2 =
n_1$.

Si $n_1$ est entier, il repr�sente le nombre de mailles sur l'axe
$Ox$. S'il est r�el, il repr�sente le pas de maillage sur l'axe
$Ox$. Par exemple, le nombre cod� \verb+1+ est entier, alors que celui
cod� \verb+1.+ est r�el (noter le point).


\begin{multicols}{2}
\psset{viewpoint=50 50 20 rtp2xyz,Decran=30}
\begin{pspicture}(-4.5,-1.5)(2.5,2.5)
\psframe(-3.5,-1.5)(2.5,2.5)
\psSolid[object=grille,
   ngrid=1,
   base=0 4 -3 3,
   linecolor=gray](0,0,0)
\axesIIID(0,0,0)(3,3,3)
\end{pspicture}

\columnbreak

\begin{verbatim}
\psSolid[object=grille,
   ngrid=1,
   base=0 4 -3 3,
   linecolor=gray](0,0,0)
\end{verbatim}
\end{multicols}

\begin{multicols}{2}
\psset{viewpoint=50 50 20 rtp2xyz,Decran=30}
\begin{pspicture}(-4.5,-1.5)(2.5,2.5)
\psframe(-3.5,-1.5)(2.5,2.5)
\psSolid[object=grille,
   ngrid=1. 1,
   base=0 4 -3 3,
   linecolor=gray](0,0,0)
\axesIIID(0,0,0)(3,3,3)
\end{pspicture}

\columnbreak

\begin{verbatim}
\psSolid[object=grille,
   ngrid=1. 1,
   base=0 4 -3 3,
   linecolor=gray](0,0,0)
\end{verbatim}
\end{multicols}


