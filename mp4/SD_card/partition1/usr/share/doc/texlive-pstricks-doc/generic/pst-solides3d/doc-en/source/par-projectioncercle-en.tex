\section{Circles}

\subsection{Direct definition}

The object \Lkeyword{cercle} allows us to define and draw a \Index{circle}.
In the \texttt{pst-solides3d} package, a circle in 2D is defined
by its centre and radius.

We use the option \Lkeyword{args} to specify the centre and radius of
the chosen circle. We can use coordinates or named variables.

The argument \texttt{\Lkeyword{range}=$t_{\rm min}$ $t_{\rm max}$} allows %$
us to specify an arc of the chosen circle.

As for all the other object, we can save the circle data using the
option \Lkeyword{name}.

\begin{LTXexample}[width=7.5cm]
\begin{pspicture}(-3,-3)(4,3.5)%
\psframe*[linecolor=blue!50](-3,-3)(4,3.5)
\psset{viewpoint=50 30 15,Decran=60}
\psset{solidmemory}
%% definition du plan de projection
\psSolid[object=plan,
   definition=equation,
   args={[1 0 0 0] 90},
   planmarks,
   name=monplan]
\psset{plan=monplan}
%% definition du point A
\psProjection[object=point,
   name=A,
   text=A,
   pos=ur](-2,1.25)
\psProjection[object=cercle,
   args=A 1,
   range=0 360]
\psProjection[object=cercle,
   args=1 1 .5,linecolor=blue,
   range=0 180]
\composeSolid
\end{pspicture}
\end{LTXexample}

\subsection{Some other definitions}

There are additional methods to define a circle in 2D. The options
\Lkeyword{definition} and \Lkeyword{args} give the following supported
methods:

\begin{itemize}

\item \texttt{\Lkeyword{definition}=\Lkeyword{ABcercle}};
\texttt{\Lkeyword{args}=$A$ $B$ $C$}.

A circle through the points $A$, $B$
and $C$.

\item \texttt{\Lkeyword{definition}=\Lkeyword{diamcercle}};
\texttt{\Lkeyword{args}=$A$ $B$}.

A circle with diameter $[AB]$.

\end{itemize}

\endinput
