\section {Trac�s d'intersections planes}

Pour chaque objet de type \textsl {solid}, il est possible de tracer
l'intersection du solide consid�r� avec un ou plusieurs plans.

L'argument num�rique \Cadre{[intersectiontype=$k$]} (valeur $-1$ par
d�faut) d�termine s'il y a ou non demande de trac�
d'intersection. Positionn� � $0$, il y a trac� des intersections.

Restent $3$ param�tres � r�gler~:

\begin{itemize}

\item \Cadre{[intersectionplan=\{$eq_1$ ... $eq_n$\}]}
d�finit la liste des �quations $e_i$ des plans de coupe. Les $e_i$
peuvent �tre �galement des objets de type plan.

\item \Cadre{[intersectionlinewidth=$w_1$ ... $w_n$]}
d�finit la liste des �paisseurs en picas $w_i$ pour chacune des 
coupes.

\item \Cadre{[intersectioncolor=$str_1$ ... $str_n$]}
d�finit la liste des couleurs des diff�rents traits de coupe.

\end{itemize}

\begin{multicols}{2}

%\begin{center}
\bgroup
\psset{unit=0.5}
\psset{lightsrc=20 -20 10,viewpoint=50 -20 10 rtp2xyz,Decran=50}
\begin{pspicture*}(-5,-4)(5,5)
\psframe(-5,-4)(5,5)
\psSolid[object=cube,
   intersectiontype=0,
   intersectionplan={[1 0 .5 2] [1 0 .5 -1]},
   intersectionlinewidth=1 2,
   intersectioncolor=(bleu) (rouge),
   RotX=20,
   RotY=90,
   RotZ=30,
   a=6,
   action=draw*,
]
\end{pspicture*}
\egroup
%\end{center}

\columnbreak

\begin{verbatim}
\psSolid[object=cube,
   intersectiontype=0,
   intersectionplan={[1 0 .5 2] [1 0 .5 -1]},
   intersectionlinewidth=1 2,
   intersectioncolor=(bleu) (rouge),
   RotX=20,RotY=90,RotZ=30,
   a=6,
   action=draw*,
]
\end{verbatim}

\end {multicols}


