\section {Cercles}

\subsection {D�finition directe}

L'objet \Cadre{cercle} permet de d�finir et tracer un cercle. 
Dans \verb+pst-solides3d+, un cercle en $2$d est d�finie par la
donn�e de son centre et de son rayon.

Sous la forme la plus simple, on utilise l'argument l'argument \verb+args+
pour sp�cifier le centre et le rayon de la droite consid�r�e. On peut
utiliser les coordonn�es ou des variables nomm�es.

L'argument \Cadre{[range=$t_{\rm min}$ $t_{\rm max}$]} permet de
sp�cifier l'intervalle de trac� du cercle consid�r�.

Comme pour les autres objets, on peut sauvegarder la
donn�e d'un cercle en utilisant l'option \Cadre{name}.

\begin{multicols}{2}
%
\begin{pspicture}(-3,-3)(4,3.5)%
\psframe*[linecolor=blue!50](-3,-3)(4,3.5)
\psset{viewpoint=50 30 15,Decran=60}
\psset{solidmemory}
%% definition du plan de projection
\psSolid[object=plan,
   definition=equation,
   args={[1 0 0 0] 90},
   planmarks,
   name=monplan,
]
\psset{plan=monplan}
%% definition du point A
\psProjection[object=point,
   name=A,
   text=A,
   pos=ur,
](-2,1.25)
\psProjection[object=cercle,
   args=A 1,
   range=0 360,
]
\psProjection[object=cercle,
   args=1 1 .5,linecolor=blue,
   range=0 180,
]
\composeSolid
\end{pspicture}
%
\columnbreak
%
\begin{gbar}
\begin{verbatim}
\psset{solidmemory}
...
\psProjection[object=point,
   name=A,text=A,pos=ur,
](-2,1.25)
\psProjection[object=cercle,
   args=A 1,range=0 360,]
\psProjection[object=cercle,
   args=1 1 .5,linecolor=blue,
   range=0 180,]
\composeSolid
\end{verbatim}
\end{gbar}
%
\end{multicols}

\subsection {Autres d�finitions}

Il existe d'autres m�thodes pour d�finir un cercle 2d. L'argument
\Cadre{definition}, coupl� � l'argument \Cadre{args} permet d'utiliser
les diff�rentes m�thodes support�es~:

\begin{itemize}

\item \Cadre {[definition=ABcercle]} ; 
\verb+args=+ $A$ $B$ $C$. Le cercle passant par les points non align�s
$A$, $B$ et $C$.

\item \Cadre {[definition=diamcercle]} ; 
\verb+args=+ $A$ $B$. Le cercle de diam�tre $[AB]$.

\end{itemize}
