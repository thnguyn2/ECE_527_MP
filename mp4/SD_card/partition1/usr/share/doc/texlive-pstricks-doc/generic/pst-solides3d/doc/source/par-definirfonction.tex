\section {D�finir une fonction}

Il est possible de d�finir des fonctions utilisables dans
l'environnement postscript. L'ensemble de d�part peut �tre $R$, $R^2$
ou $R^3$, et l'ensemble d'arriv�e peut �tre $R$, $R^2$ ou $R^3$.

La d�finition se fait avec la macro \verb!\defFunction!. Cette
macro n�cessite six arguments, dont un seul est optionnel.

\verb!\defFunction[<options>]{<nom>}(<var>){<x(var)>}{<y(var)>}{<z(var)>}!

\begin{table}[h]
\begin{tabular}{p{2cm}p{11cm}}
\verb!<options>! & On y ins�re les option typiques de PSTricks, comme
\verb!linewidth! etc., et en plus, quelques unes d�finies par
\verb!pst-solides3d!. Une tr�s charmante option est \verb!algebraic!,
avec quelle on peut �viter la notation RPN (Reverse Polish
Notation). Toutes options sont des paires (cl�,valeur) et sont s�par�es
avec des virgules.\\

\verb!<nom>! & C'est un nom unique de votre choix -- mais attention:
�vitez des noms avec des accents, PostScript ne les aime pas du
tout.\\

\verb!<var>! & On y ins�re au maximum trois variables arbitraires,
s�par�es avec des virgules. \\

\verb!<x(var)>! \verb!<y(var)>! \verb!<z(var)>! & On y met des
fonctions d�pendant des variables d�finies pour les directions
euclidienness $x,\,y,\,z$. Si une de ces trois directions n'est pas
voulue, ins�rez  un 0 entre les parenth�ses -- ce qui vous donne la
possibilit� de d�finir aussi des projet�s plans de courbes de fonctions.
\end{tabular}
\end{table}

Quand vous avez d�fini une fonction, cette fonction est toujours
reprise avec son \verb!<nom>! choisi.

Voil\`{a} quelques exemples:
\begin{itemize}
    \item \verb!\defFunction{moncercle}(t){t cos 3 mul}{0}{t sin 3 mul}!

    donne un cercle de rayon 3 dans le plan $xOz$ (notation RPN).
    \item \verb!\defFunction[algebraic]{helice}(t){cos(t)}{sin(t)}{t}!

    donne une h�lice  en notation alg�brique.

    \item \verb!\defFunction[algebraic]{F}(t){t}{}{}!
    donne une fonction de \textbf{R} dans \textbf{R}

    \item \verb!\defFunction[algebraic]{F}(t){t}{}{}!
    donne une fonction de \textbf{R} dans \textbf{R$^{\textbf 2}$}

    \item \verb!\defFunction[algebraic]{F}(t){t}{t}{t}!
    donne une fonction de \textbf{R} dans \textbf{R$^{\textbf 3}$}

\end{itemize}

\llap {\dbend } Il nous reste encore du travail � faire sur cette
macro, et elle ne permet pour le moment pas de choisir des noms
de variables quelconques, car ils risquent d'entrer en conflit avec
des noms d�j� existant. Merci d'utiliser des noms analogues � ceux
utilis�s dans la documentation. Une bonne strat�gie consiste �
utiliser syst�matiquement un ou plusieurs caract�res num�riques � la
fin de vos noms de variables.
