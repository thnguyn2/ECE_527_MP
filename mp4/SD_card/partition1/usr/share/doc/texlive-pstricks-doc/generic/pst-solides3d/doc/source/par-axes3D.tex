\section{Les axes en 3D}

la commande \Cadre{\textbackslash axesIIID[options](x1,y1,z1)(x2,y2,z2)} trace les axes $Ox$,
$Oy$ et $Oz$ en pointill�s de $O$ respectivement, jusqu'au point de coordonn�es
$(x_1,0,0)$ pour l'axe des $x$, $(0,y_1,0)$ pour l'axe des $y$ et
$(0,0,z_1)$ pour l'axe des $z$ et ensuite en trait continu jusqu'aux
points $(x_2,0,0)$, $(0,y_2,0)$ et $(0,0,z_2)$.

Les options sont les suivantes :
\begin{itemize}
    \item toutes les options de couleur, d'�paisseur du trait, ainsi que des caract�ristiques des fl�ches.
    \item \Cadre{labelsep=valeur} qui permet de placer � la distance souhait�e de l'extr�mit� de la fl�che, l'�tiquette de l'axe, sa valeur par d�faut est \Cadre{labelsep=5pt}, il s'agit de la distance r�elle en trois dimensions et non sur l'�cran.
    \item Le choix des �tiquettes(\textit{labels}) de chaque axe avec l'option \Cadre{axisnames={a,b,c}}, avec par d�faut \Cadre{axisnames={x,y,z}}.
    \item La possibilit� de sp�cifier, le style de ces �tiquettes avec l'option : \Cadre{axisemph={\boldmath\Large\color{red}}}, par d�faut il n'y a pas de style pr�d�fini, c'est-�-dire que si l'on ne pr�cise rien on aura \Cadre{$x,y,z$}.
    \item \Cadre{showOrigin} est un bool�en, \texttt{true} par d�faut, s'il est positionn� � \Cadre{showOrigin=false}  les pointill�s ne seront plus trac�s depuis l'origine.
    \item \Cadre{mathLabel} est un bool�en, \texttt{true} par d�faut, qui dans ce cas �crit les �tiquettes en mode math�matique, positionn� � \Cadre{mathLabel=false} on passe dans le mode usuel.
\end{itemize}
\encadre{Les �tiquettes sont plac�es aux extr�mit�s des axes dans leur prolongement.}

\begin{LTXexample}[width=5cm]
\begin{pspicture}(-2,-2)(3,3)
\psset{viewpoint=100 30 20,Decran=100}
\psframe(-2,-2)(3,3)
\psSolid[object=cube,a=2,
        action=draw*,
        fillcolor=magenta!20]
\axesIIID[showOrigin=false](1,1,1)(3,2,2.5)
\end{pspicture}
\end{LTXexample}

\begin{LTXexample}[width=5cm]
\begin{pspicture}(-2,-1)(3,4)
\psset{viewpoint=100 45 20,Decran=100}
\psframe(-2,-1)(3,4)
\psSolid[object=cylindre,h=2,r=1,
        action=draw*,mode=4,
        fillcolor=green!20]
\axesIIID[linewidth=1pt,linecolor=red,arrowsize=5pt,
          arrowinset=0,axisnames={a,b,c},
          axisemph={\boldmath\Large\color{red}},
          labelsep=10pt]
         (1,1,2)(2,2,3)
\end{pspicture}
\end{LTXexample}