\section{The object \texttt{point}}

\subsection{Definition via coordinates}

The object \Lkeyword{point} defines a \Index{point}. The simplest method is to use the argument \texttt{\Lkeyword{args}=$x$ $y$ $z$} to specify its coordinates.
If we have already named a point $M(x, y, z)$ (see chapter ``\textit{Advanced usage\/}''), we can easily use the argument \texttt{args=$M$}.

\subsection{Some other definitions}

There are some other possibilities for defining a point. Here a list of possible definitions with the appropriate arguments:

\begin{itemize}

\item \texttt{\Lkeyword{definition}=\Lkeyval{solidgetsommet}};
\texttt{\Lkeyword{args}= $solid$ $k$}.

The vertex with index $k$ of the solid $solid$.

\item \texttt{\Lkeyword{definition}=\Lkeyval{solidcentreface}};
\texttt{\Lkeyword{args}=$solid$ $k$}.

The centre of the face with index $k$ of the solid $solid$.

\item \texttt{\Lkeyword{definition}=\Lkeyval{isobarycentre3d}};
\texttt{\Lkeyword{args}=\{[ $A_0$ $\ldots $ $A_{n}$ ]\}}. 

   {The isobarycentre of the system $[(A_0, 1);
   \ldots ; (A_n, 1)]$.}

\item \texttt{\Lkeyword{definition}=\Lkeyval{barycentre3d}};
\Lkeyword{args}=  \{[ $A$ $a$ $B$ $b$ ] \}.

   {The barycentre of the system $[(A, a) ; (B, b)]$.}

\item \texttt{\Lkeyword{definition}=\Lkeyval{hompoint3d}};
\texttt{\Lkeyword{args}={$M$ $A$ $\alpha $}}.

   {The image of $M$ via a homothety with centre $A$ and ratio $\alpha $.}

\item \texttt{\Lkeyword{definition}=\Lkeyval{sympoint3d}};
\texttt{\Lkeyword{args}= {$M$ $A$}}.

   {The image of $M$ via the center of symmetry $A$}%I don't understand

\item \texttt{\Lkeyword{definition}=\Lkeyval{translatepoint3d}};
\texttt{\Lkeyword{args}=   {$M$ $u$}}.

   {The image of $M$ under the translation via the vector $\vec u$}

\item \texttt{\Lkeyword{definition}=\Lkeyval{scaleOpoint3d}};
\texttt{\Lkeyword{args}= {$x$ $y$ $z$  $k_1$ $k_2$ $k_3$}}.

   {This gives a ``dilation'' \ of the coordinates of the point $M (x, y,
   z)$ on the axes $Ox$, $Oy$ and $Oz$ each multiplied by an appropriate factor $k_1$,
   $k_2$ and $k_3$}

\item \texttt{\Lkeyword{definition}=\Lkeyval{rotateOpoint3d}};
\texttt{\Lkeyword{args}= {$M$ $\alpha_x$ $\alpha_y$ $\alpha_z$}}.

   {The image of $M$ through consecutive rotations---centered at $O$---and with respective angles
   $\alpha_x$, $\alpha_y$ and $\alpha_z$ around the axes $Ox$,
   $Oy$ and $Oz$.}



%% Projection orthogonale d'un point 3d sur un plan
%% Mx My Mz (=le point a projeter)
%% Ax Ay Az (=un point du plan)
%% Vx Vy Vz (un vecteur normal au plan)
\item \Lkeyword{definition}=\Lkeyval{orthoprojplane3d};
\texttt{\Lkeyword{args}= {$M$ $A$ $\vec v$}}.

   {The projection of the point $M$ to the plane $P$ which is defined
   by the point $A$ and the vector $\vec v$, perpendicular to $P$.}

\item \texttt{\Lkeyword{definition}=\Lkeyval{milieu3d}};
\texttt{\Lkeyword{args}= {$A$ $B$}}.

   {The midpoint of $[AB]$}

\item \texttt{\Lkeyword{definition}=\Lkeyval{addv3d}};
\texttt{\Lkeyword{args}= {$A$ $u$}}.

   {Gives the point $B$ so that $\overrightarrow {AB} = \vec u$}

\end{itemize}

\endinput
