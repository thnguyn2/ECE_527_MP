\section {Angle droit}

L'objet \Cadre{rightangle} permet de d�finir un angle droit. Sa
syntaxe est 
\Cadre{[object=rightangle,args=$A$ $B$ $C$]}

\begin{multicols}{2}

%% projete orthpgonal d'un point sur une droite
%%
\begin{pspicture}(-3,-2.5)(3.5,2.5)%
\psframe*[linecolor=blue!50](-3,-2.5)(3.5,2.5)
\psset{lightsrc=viewpoint,viewpoint=50 30 15,Decran=40}
\psset{solidmemory}
%% definition du plan de projection
\psSolid[object=plan,
   definition=equation,
   args={[1 0 1 0] 90},
   base=-4 4 -3 3,
   fillcolor=white,
   linecolor=gray!30,
%   plangrid,
   planmarks,
   name=monplan,
]
\psset{plan=monplan,visibility=false}
%% definition droite d
\psProjection[object=droite,
   definition=horizontale,
   args=-1,
   name=d,
]
\psset{fontsize=15}
%% definition du point M
\psProjection[object=point,
   args=-2 1,
   name=M,
   text=M,
   pos=ul,
]
%% definition du point H
\psProjection[object=point,
   definition=orthoproj,
   args=M d,
   name=H,
   text=H,
   pos=dr,
]
%% definition du point H' pour orienter l'angle droit
%% et mettre la legende
\psProjection[object=point,
   definition=xdpoint,
   args=2 d,
   name=H',
   action=none,
   text=d,
   pos=ur,
]
%% definition d'une ligne 
\psProjection[object=line,
   args=M H,
]
%% dessin angle droit
\psProjection[object=rightangle,
   args=M H H',
]
\composeSolid
%\axesIIID(4,4,2)(5,5,6)
\end{pspicture}
\columnbreak

\begin{gbar}
\begin{verbatim}
\psProjection[object=droite,
   definition=horizontale,args=-1,name=d,]
\psProjection[object=point,args=-2 1,
   name=M,text=M,pos=ul,]
\psProjection[object=point,
   definition=orthoproj,
   args=M d,name=H,text=H,pos=dr,]
%% definition du point H' pour orienter l'angle droit
\psProjection[object=point,
   definition=xdpoint,args=2 d,
   name=H',action=none,text=d,pos=ur,]
\psProjection[object=line,args=M H,]
\psProjection[object=rightangle,args=M H H',]
\end{verbatim}
\end{gbar}

\end{multicols}

