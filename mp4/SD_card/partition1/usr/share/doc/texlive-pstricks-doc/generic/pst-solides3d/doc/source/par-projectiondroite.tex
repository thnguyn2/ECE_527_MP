\section {Droites}

\subsection {D�finition directe}

L'objet \Cadre{droite} permet de d�finir et tracer une droite. 
Dans \verb+pst-solides3d+, une droite en $2$d est d�finie par la
donn�e de $2$ de ses points.

Sous la forme la plus simple, on utilise l'argument l'argument \verb+args+
pour sp�cifier $2$ points de la droite consid�r�e. On peut utiliser
les coordonn�es ou des points nomm�s.

Comme pour les points et les vecteurs, on peut sauvegarder la
donn�e d'une droite en utilisant l'option \Cadre{name}.

\begin{multicols}{2}
%
\begin{pspicture}(-3,-3)(4,3.5)%
\psframe*[linecolor=blue!50](-3,-3)(4,3.5)
\psset{viewpoint=50 30 15,Decran=60}
\psset{solidmemory}
%% definition du plan de projection
\psSolid[object=plan,
   definition=equation,
   args={[1 0 0 0] 90},
   planmarks,
   name=monplan,
]
\psset{plan=monplan}
%% definition du point A
\psProjection[object=point,
   name=A,
   text=A,
   pos=ur,
](-2,1.25)
\psProjection[object=point,
   name=B,
   text=B,
   pos=ur,
](1,.75)
\psProjection[object=droite,
   linecolor=blue,
   args=0 0 1 .5,
]
\psProjection[object=droite,
   linecolor=orange,
   args=A B,
]
\composeSolid
\end{pspicture}
%
\columnbreak
%
\begin{gbar}
\begin{verbatim}
\psProjection[object=point,
   name=A,text=A,pos=ur,](-2,1.25)
\psProjection[object=point,
   name=B,text=B,pos=ur,](1,.75)
\psProjection[object=droite,
   linecolor=blue,args=0 0 1 .5,]
\psProjection[object=droite,
   linecolor=orange,args=A B,]
\end{verbatim}
\end{gbar}
%
\end{multicols}

\subsection {Autres d�finitions}

Il existe d'autres m�thodes pour d�finir une droite 2d. L'argument
\Cadre{definition}, coupl� � l'argument \Cadre{args} permet d'utiliser
les diff�rentes m�thodes support�es~:

\begin{itemize}

\item \Cadre {[definition=horizontale]} ; 
\verb+args=+ $b$. La droite d'�quation $y=b$.

\item \Cadre {[definition=verticale]} ; 
\verb+args=+ $a$. La droite d'�quation $x=a$.

\item \Cadre {[definition=paral]} ; 
\verb+args=+ $d$ $A$. La droite parall�le � la droite $d$ passant par
le point $A$.

\item \Cadre {[definition=perp]} ; 
\verb+args=+ $d$ $A$. La droite perpendiculaire � la droite $d$ passant par
le point $A$.

\item \Cadre {[definition=mediatrice]} ; 
\verb+args=+ $A$ $B$. La droite m�diatrice du segment $[AB]$.

\item \Cadre {[definition=bissectrice]} ; 
\verb+args=+ $A$ $B$ $C$. La droite bissectrice de l'angle $\widehat
{ABC}$. 

\item \Cadre {[definition=axesymdroite]} ; 
\verb+args=+$d$ $D$. Sym�trique de la droite $d$ par rapport � la
droite $D$.

\item \Cadre {[definition=rotatedroite]} ; 
\verb+args=+$d$ $I$ $r$. Image de la droite $d$ par la rotation de
centre $I$ et d'angle $r$ (en degr�s)

\item \Cadre {[definition=translatedroite]} ; 
\verb+args=+$d$ $u$. Image de la droite $d$ par la translation de
vecteur $\vec u$.

\end{itemize}
