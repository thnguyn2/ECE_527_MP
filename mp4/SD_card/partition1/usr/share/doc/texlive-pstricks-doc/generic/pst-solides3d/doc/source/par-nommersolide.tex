\section {Nommer un solide}

Pour certaines utilisations, on a besoin de stocker un solide en
m�moire afin de pouvoir y faire r�f�rence par la suite. Pour ce faire
on dispose du bool�en \verb+solidmemory+, qui permet la transmission
d'une variable tout au long de la sc�ne.

En revanche, l'activation de ce bool�en d�sactive le dessin imm�diat
des macros \verb+\psSolid+, \verb+\psSurface+ et
\verb+\psProjection+. Pour obtenir ce dessin, on utilise la macro
\verb+\composeSolid+ � la fin de la sc�ne.

Lorsque l'activation \Cadre{$\backslash $psset\{solidmemory\}} est faite, on peut
alors utiliser l'option \Cadre{[name=...]} de la macro \verb+\psSolid+.

Dans l'exemple ci-dessous, on construit un solide color�, que l'on
sauvagarde sous le nom $A1$. On le dessine ensuite, apr�s coup, en
utilisant l'objet \Cadre{[object=load]} avec le param�tre
\Cadre{[load=$name$]}. 

\`A noter que l'instruction \verb+linecolor=blue+ utilis�e lors de la
construction de notre cube n'a pas d'impact sur le dessin~: seule la
structure du solide a �t� sauvegard� (sommets, faces, couleurs des
faces), pas l'�paisseur de la ligne de trac� ou sa couleur ou la
position de la source lumineuse. C'est au moment du dessin du solide
consid�r� qu'il faut r�gler ces param�tres.

Enfin, on remarquera l'utilisation de l'option
\Cadre{[deactivatecolor]} qui permet au cube de garder sa couleur
rouge d'origine (sinon les couleurs par d�faut auraient repris le
dessus dans l'objet \verb+load+).

\begin{multicols}{2}
\bgroup
\psset{unit=0.75}
\psset{lightsrc=10 0 10,viewpoint=50 -20 10 rtp2xyz,Decran=50}
\begin{pspicture*}(-4,-4)(5,4)
\psframe(-4,-4)(5,4)
\psset{solidmemory}
\psSolid[object=cube,
      linecolor=blue,
      a=4,fillcolor=red!50,
      ngrid=3,
      action=none,
      name=A,
      ](0,0,0)
\psSolid[object=load,
   deactivatecolor,
   load=A]
\composeSolid
\end{pspicture*}
\egroup

\columnbreak

\begin{verbatim}
\psset{solidmemory}
\psSolid[object=cube,
      linecolor=blue,
      a=4,fillcolor=red!50,
      ngrid=3,
      action=none,
      name=A,
      ](0,0,0)
\psSolid[object=load,
   deactivatecolor,
   load=A]
\composeSolid
\end{verbatim}
\end{multicols}


\llap {\dbend } 
Avec l'option \verb+solidmemory+, les noms de variables sont
relativement bien encapsul�s, et il n'y a pas de conflit avec les
variables de dvips par exemple. Il reste par contre le risque de
surcharge des noms utilis�s par \verb+solides.pro+. On peut utiliser
tous les noms de variables � un seul carct�re alphab�tique, mais il
faut �viter d'utiliser des noms comme 
\verb+vecteur+,
\verb+distance+,
\verb+droite+, etc\dots qui sont d�j� d�finis par le package.
