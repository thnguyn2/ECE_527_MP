\section {Tronquer les sommets d'un solide}

L'option \Cadre{[trunc]} permet de tronquer les sommets soit
globalement, soit individuellement. Cette option utilise l'argument
\Cadre{[trunccoeff]} (valeur $0,25$ par d�faut) qui indique le rapport
$k$ utiliser pour la troncature ($0<k\leq 0,5$).
%
\begin{itemize}
  \item \Cadre{[trunc=all]} tronque tous les sommets ;
  \item \Cadre{[trunc=0 1 2 3]} tronque les sommets \texttt{[0,1,2 et 3]} ;
\end{itemize}
%
\begin{multicols}{2}
\psset{unit=0.5}
\setlength{\columnseprule}{1pt}
\centerline{
\begin{pspicture}(-5,-5)(5,5)
\psframe(-5,-5)(5,5)
\psset{Decran=20}
\psSolid[
   action=draw,
   object=cube,
   RotZ=30,
   trunccoeff=.2,trunc=all,
]%
\end{pspicture}}
\begin{verbatim}
\psSolid[object=cube,
   action=draw,RotZ=30,
   trunccoeff=.2,trunc=all,
]%
\end{verbatim}
\columnbreak
\centerline{
\begin{pspicture}(-5,-5)(5,5)
\psframe(-5,-5)(5,5)
\psset{Decran=20}
\psSolid[action=draw,
   object=cube,
   RotZ=30,
   trunccoeff=.2,
   trunc=0 1 2 3,
]%
\end{pspicture}}
\begin{verbatim}
\psSolid[object=cube,
   RotZ=30,action=draw,
   trunccoeff=.2,
   trunc=0 1 2 3,
]%
\end{verbatim}
\end{multicols}
%\newpage
