\section {Point}

\subsection {D�finition � partir des coordonn�es}
 
L'objet \Cadre{point} permet de d�finir un point. Sous sa forme la
plus simple, on utilise l'argument \Cadre{[args=$x$ $y$ $z$]} pour
en sp�cifier les coordonn�es. Si on a pr�c�demment nomm� $M$ un point
$(x, y, z)$ (voir chapitre \textsl{Utilisation avanc�e\/}), on peut
utiliser l'argument \Cadre{[args=$M$]}.

\subsection {Autres modes de d�finition}

Il existe d'autres possibilit�s pour d�finir un point. Voici une
liste des d�finitions possibles avec les arguments correspondant~:

\begin{itemize}

\item \Cadre {[definition=solidgetsommet]} ; 
\verb+args=+ $solik$ $k$.
Le sommet d'indice $k$ du solid $solid$.

\item \Cadre {[definition=solidcentreface]} ; 
\verb+args=+ $solik$ $k$.
Le centre de la face d'indice $k$ du solid $solid$.

\item
\Cadre {[definition=isobarycentre3d]}
\verb+args=+
   {\{$[$ $A_0$ $\ldots $ $A_{n}$ $]$\}}    
   {le barycentre du syst�me $[(A_0, 1) ;
   \ldots ; (A_n, 1)]$}

\item
\Cadre {[definition=barycentre3d]}
\verb+args=+
   {\{$[$ $A$ $a$ $B$ $b$ $]$\}}    
   {le barycentre du syst�me $[(A, a) ; (B, b)]$}

\item
\Cadre {[definition=hompoint3d]}
\verb+args=+
   {$M$ $A$ $\alpha $}
   {l'image de $M$ par l'homoth�tie de centre $A$ et de
   rapport $\alpha $}

\item
\Cadre {[definition=sympoint3d]}
\verb+args=+
   {$M$ $A$}
   {l'image de $M$ par la sym�trie de centre $A$}

\item
\Cadre {[definition=translatepoint3d]}
\verb+args=+
   {$M$ $u$}
   {l'image de $M$ par la translation de vecteur $\vec u$}

\item
\Cadre {[definition=scaleOpoint3d]}
\verb+args=+
   {$x$ $y$ $z$  $k_1$ $k_2$ $k_3$}
   {op�re une \og dilatation\fg \ des coordonn�es du point $M (x, y,
   z)$ sur les axes $Ox$, $Oy$ et $Oz$ suivant les facteurs $k_1$,
   $k_2$ et $k_3$}

\item
\Cadre {[definition=rotateOpoint3d]}
\verb+args=+
   {$M$ $\alpha_x$ $\alpha_y$ $\alpha_z$}
   {l'image de $M$ par les rotations successives de centre $O$ et d'angles
   respectifs $\alpha_x$ $\alpha_y$ $\alpha_z$ sur les axes $Ox$,
   $Oy$, $Oz$}



%% Projection orthogonale d'un point 3d sur un plan
%% Mx My Mz (=le point a projeter) 
%% Ax Ay Az (=un point du plan) 
%% Vx Vy Vz (un vecteur normal au plan)
\item
\Cadre {[definition=orthoprojplane3d]}
\verb+args=+
   {$M$ $A$ $\vec v$}
   {Le projet� du point $M$ sur le plan $P$ d�fini
   par le point $A$ et le vecteur $\vec v$, normal � $P$.}

\item
\Cadre {[definition=milieu3d]}
\verb+args=+
   {$A$ $B$}
   {Le milieu de $[AB]$}

\item
\Cadre {[definition=addv3d]}
\verb+args=+
   {$A$ $u$}
   {Le point $B$ tel que $\overrightarrow {AB} = \vec u$}

\end{itemize}


