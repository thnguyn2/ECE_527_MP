\section {Num�roter les facettes}

L'option \verb+numfaces+ permet d'afficher sur chaque face son indice
rorrespondant. 
\begin{itemize}
  \item \Cadre{[numfaces=all]} affiche tous les num�ros de faces ;
  \item \Cadre{[numefaces=0 1 2 3]} affiche tous les num�ros de faces
  \texttt{[0,1,2 et 3]} ; 
\end{itemize}
L'option \Cadre{fontsize} permet de fixer la taille de la police
utilis�e.
Enfin, le bool�en \Cadre{visibility} permet de sp�cifier si on doit ou
non afficher le num�ro de face si la face n'est pas visible. Par
d�faut, on a \verb+visibility=true+, et on tient compte de la
visibilit� (ie. num�ro pas affich� si la face n'est pas visible)

\begin{multicols}{2}
\psset{unit=1}
\psset{viewpoint=50 20 30 rtp2xyz,Decran=50}
\begin{pspicture}(-4,-3)(3,1.5)
\psframe(-4,-3)(3,1.5)
\psSolid[object=grille,
   base=0 4 -2 2,
   numfaces=2 6 7 10,
   linecolor=gray](0,0,0)
\axesIIID(0,0,0)(4,2,1)
\end{pspicture}

\columnbreak

\begin{verbatim}
\psSolid[object=grille,
         base=0 4 -2 2,
         numfaces=2 6 7 10,
         linecolor=gray](0,0,0)
\end{verbatim}
\end{multicols}

%% \begin{multicols}{2}
%% 
%% \bgroup
%% \psset{viewpoint=50 20 30 rtp2xyz}
%% \begin{center}
%% \psset{unit=0.75}
%% \psset{lightsrc=30 -20 10,viewpoint=50 -20 10 rtp2xyz,Decran=50}
%% \begin{pspicture*}(-5,-4)(6,6)
%% \psframe(-5,-4)(6,6)
%% \axesIIID(0,0,0)(4,4,4)
%% \psSolid[object=cube,
%%    RotY=90,
%%    ngrid=4,
%%    numfaces=2 6 10,
%%    action=draw**](0,0,0)
%% \end{pspicture*}
%% \end{center}
%% \egroup
%% 
%% \columnbreak
%% 
%% \begin{verbatim}
%% \axesIIID(0,0,0)(4,4,4)
%% \psSolid[object=cube,
%%    RotY=90,
%%    ngrid=4,
%%    numfaces=2 6 10,
%%    action=draw**](0,0,0)
%% \end{verbatim}
%% 
%% \end{multicols}

\begin{multicols}{2}
\bgroup
\psset{unit=0.75}
\psset{viewpoint=50 -20 10 rtp2xyz,Decran=50}
\begin{pspicture*}(-4,-3)(4,3)
\psframe(-4,-2.9)(4,3)
\psSolid[object=cube,
   RotY=90,
   ngrid=4,
   fontsize=15,
   action=draw,
   numfaces=all](0,0,0)
\end{pspicture*}
\egroup
\columnbreak

\begin{verbatim}
\psSolid[object=cube,
         RotY=90,
         ngrid=4,
         fontsize=15,
         action=draw,
         numfaces=all](0,0,0)
\end{verbatim}
\end{multicols}

%%% exemple 3

Les options de \verb+\psSolid+ acceptent des commandes postcript, et
en particulier les boucles \verb+for+. 

Ainsi l'instruction \verb+[numfaces=0 1 5 {} for]+ demande la
num�rotation de toutes les faces dont l'indice est compris entre $0$
et $5$. L'instruction \verb+[numfaces=8 3 23 {} for]+ demande la
num�rotation d'une face sur $3$ entre les indices $8$ et $23$.

\begin{multicols}{2}
\bgroup
\psset{unit=0.75}
\psset{viewpoint=50 -20 10 rtp2xyz,Decran=50}
\begin{pspicture*}(-4,-3)(4,3)
\psframe(-4,-3)(4,3)
\axesIIID(0,0,0)(8,3,2)
\psSolid[object=grille,
   RotY=90,
   RotZ=180,
   ngrid=1.,
   fontsize=15,
   numfaces=
      0 1 5 {} for
      8 3 23 {} for,
   base=-2 2 -3 3,
   visibility=false,
   action=draw](0,0,0)
\end{pspicture*}
\egroup

\columnbreak

\begin{verbatim}
\axesIIID(0,0,0)(8,3,2)
\psSolid[object=grille,
         RotY=90,
         RotZ=180,
         ngrid=1.,
         fontsize=15,
         numfaces=
            0 1 5 {} for
            8 3 23 {} for,
         base=-2 2 -3 3,
         visibility=false,
         action=draw](0,0,0)
\end{verbatim}
\end{multicols}
