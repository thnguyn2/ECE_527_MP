\section {Vecteurs}

\subsection {D�finition � partir des coordonn�es}
 
L'objet \Cadre{vecteur} permet de d�finir un vecteur. Sous sa forme la
plus simple, on utilise l'argument \Cadre{[args=$x$ $y$ $z$]} pour
en sp�cifier les coordonn�es.

\begin{multicols}{2}

\psset{lightsrc=10 -20 50,viewpoint=50 -20 30 rtp2xyz,Decran=100}
\begin{pspicture*}(-1,-1)(1,2)
\psframe(-1,-1)(1,2)
\psSolid[object=vecteur,
action=draw*,
   args=0 0 1,
   linecolor=yellow]%
\psSolid[object=vecteur,
   args=1 0 0,
   linecolor=red]
\psSolid[object=vecteur,
   args=0 0 1,
   linecolor=blue](1,0,0)
\end{pspicture*}

\columnbreak

\begin{verbatim}
\psSolid[object=vecteur,
   args=0 0 1,
   linecolor=yellow]%
\psSolid[object=vecteur,
   args=1 0 0,
   linecolor=red]
\psSolid[object=vecteur,
   args=0 0 1,
   linecolor=blue](1,0,0)
\end{verbatim}
\end{multicols}

\subsection {D�finition � partir de 2 points}

On peut �galement d�finir un vecteur par la donn�e de 2 points $A$ et
$B$ de $R^3$. On utilise alors les arguments
\Cadre{[definition=vecteur3d]} et \Cadre{[args=$x_A$ $y_A$ $z_A$ $x_B$
$y_B$ $z_B$]} o�  $(x_A, y_A, z_A)$ et $(x_B, y_B, z_B)$  sont les
coordonn�es respectives des points $A$ et $B$

Si les points $A$ et $B$ ont �t� pr�alablement d�finis, alors on peut
utiliser des variables nomm�es~: 
\Cadre{[args=$A$ $B$]}.

\begin{multicols}{2}

\psset{lightsrc=10 -20 50,viewpoint=10 -10 10,Decran=20}
\begin{pspicture*}(-3,-3)(4.5,2)
\psframe(-3,-3)(4.5,2)
\psSolid[object=plan,
   linecolor=gray,
   definition=equation,
   args={[0 0 1 0]},
   base=-1 3 -2 2,
   planmarks,
   plangrid,
]
\psSolid[object=vecteur,
   definition=vecteur3d,
   args=0 0 1 1 1 1]%
\end{pspicture*}

\columnbreak

\begin{verbatim}
\psSolid[object=vecteur,
   definition=vecteur3d,
   args=0 0 1 1 1 1]%
\end{verbatim}
\end{multicols}

\subsection {Autres modes de d�finition}

Il existe d'autres possibilit�s pour d�finir un vecteur. Voici une
liste des d�finitions possibles avec les arguments correspondant~:

\begin{itemize}

\item \Cadre {[definition=addv3d]} ; 
\verb+args=+ $\vec u$ $\vec v$.
addition de 2 vecteurs.

\item \Cadre {[definition=subv3d]} ; 
\verb+args=+ $\vec u$ $\vec v$.
diff�rence de 2 vecteurs. 

\item \Cadre {[definition=mulv3d]} ; 
\verb+args=+ $\vec u$ $\lambda $.
multiplication d'un vecteur par un r�el.

\item \Cadre {[definition=vectprod3d]} ; 
\verb+args=+ $\vec u$ $\vec v$.
produit vectoriel de 2 vecteurs.

\item \Cadre {[definition=normalize3d]} ; 
\verb+args=+ $\vec u$.
Renvoie le vecteur $\Vert \vec u\Vert ^{-1} \vec u$.

\end{itemize}
