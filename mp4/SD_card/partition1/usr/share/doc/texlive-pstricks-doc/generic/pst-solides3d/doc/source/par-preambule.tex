\section {Avant-propos}

Le package pr�sent� dans ce document est issu d'un travail
collaboratif initi� sur la liste de diffusion du site syracuse
(\url{http://melusine.eu.org/syracuse}).

L'id�e est n�e de la confrontation des travaux de Jean-Paul Vignault
sur le logiciel jps2ps%
\footnote{\url{http://melusine.eu.org/syracuse/bbgraf/}}
avec ceux de Manuel Luque sur PSTricks%
\footnote{\url{http://melusine.eu.org/syracuse/pstricks/pst-v3d/}}, en
particulier dans le domaine de la repr�sentation des solides dans une
sc�ne en 3d.

Les deux auteurs ont d�cid� d'unifier leurs efforts dans l'�criture
d'un package PSTricks d�di� � la repr�sentation de sc�nes 3d. Le
travail s'effectue sur la machine \textsl{melusine}, dans un
environnement informatique pr�par� et maintenu par Jean-Michel Sarlat.

L'�quipe s'est ensuite �toff�e avec l'arriv�e d'Arnaud Schmittbuhl
et de J�rgen Gilg, ce dernier s'�tant sp�cialis� dans le beta-test �
base d'animations%
\footnote{\url{http://melusine.eu.org/syracuse/pstricks/pst-solides3d/animations/}}.
